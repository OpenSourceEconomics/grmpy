\begin{frame}\begin{center}
\LARGE\textbf{Setup}
\end{center}\end{frame}



\begin{frame}
\textbf{Setup}

\medskip
\begin{itemize}\setlength\itemsep{1em}
\item Normal linear-in-parameters version of the generalized Roy model.
\end{itemize}
\begin{align*}
\text{Potential Outcomes} &\qquad \\
Y_1 = \beta_1 X + U_1      &\qquad Y_0 = \beta_0 X + U_0 \\
    & \\
\text{Observed Outcomes}  &\qquad \text{Choice} \\
Y = D Y_1 + (1 - D)Y_0 &\qquad D = \mathrm{I}[D^{*} > 0] \\
                       &\qquad D^{*} = \mu_D(X, Z) - V \\
                       &\qquad U_D = F_V(V)
\end{align*}
The unobservables follow a normal distribution $(U_1, U_0, V) \sim N(0, \Sigma)$ with mean zero and covariance matrix $\Sigma$. 
\end{frame}

\begin{frame}\begin{center}
\LARGE\textbf{Features}
\end{center}\end{frame}

\begin{frame}
\textbf{Features}

\medskip
\begin{itemize}\setlength\itemsep{1em}
\item \textit{grmpy} is currently capable of the following features:
\begin{itemize}\setlength\itemsep{1em}
  \item Simulating a dataset based on your own specifications.
  \item Providing some useful information about the simulated dataset for instance:
  
  \medskip
    \begin{itemize}\setlength\itemsep{1em}
    \item Distributional outcome characteristics
    \item $B^{ATE}, B^{TT}, B^{TUT}$
    \item $B^{MTE}$ by ventile
    \end{itemize}
  \item Estimating the coefficients of interest given a dataset (of a specific form).
\end{itemize}
\end{itemize}

\end{frame}

\begin{frame}
\textbf{Install the package}

\medskip
\begin{itemize}\setlength\itemsep{1em}
\item OS, Linux : Use the pip install manager (\textit{pip install grmpy}) or download the package via \href{https://github.com/grmToolbox/grmpy}{GitHub} and install it manually.
\item Windows:  The same procedure as for Linux, OS but you have to verify that the \textit{numpy} package is already installed on your machine.
\end{itemize}
\end{frame}

\begin{frame}
\textbf{Initialization file}

\medskip
\begin{itemize}\setlength\itemsep{1em}
\item The initialization file provides the user with the opportunity to specify all parameters of his/her model, for instance:\medskip
  \begin{itemize}\setlength\itemsep{1em}
  \item Simulation parameters (number of observations, name of the output files)
  \item Estimation parameters (optimization algorithm, start values)
  \item Optimizer spezifications
  \item Coefficients and covariance parameters, dummy variables ...
  \end{itemize}
\end{itemize}
\end{frame}

\begin{frame}
\textbf{Initialization file}

\medskip
\begin{itemize}
\item \href{examples/tutorial.grmpy.ini}{Example}
\item for a detailed explanation see: \href{http://grmpy.readthedocs.io/en/latest/tutorial.html}{\textit{grmpy}-documentation}
\end{itemize}
\vfill
\end{frame}

\begin{frame}
\textbf{Simulation}

\medskip
\begin{itemize}\setlength\itemsep{1em}
\item \textit{grmpy.simulate():}

\medskip
\begin{itemize}\setlength\itemsep{1em}
\item Input: path of the initialization file.
\item The function returns a data frame based on your specifications and different output files.

\medskip
\begin{itemize}\setlength\itemsep{1em}
\item The data set as a pickle and a txt file.
\item An \href{examples/data.grmpy.info}{Info file} that provides the distributional characteristics of the data as well as information about the different treatment effects.
\end{itemize}

\end{itemize}
\end{itemize}
\end{frame}

\begin{frame}
\textbf{Estimation}

\medskip
\begin{itemize}\setlength\itemsep{1em}
\item \textit{grmpy.fit():}\medskip
  \begin{itemize}\setlength\itemsep{1em}
  \item Input: path of the initialization file.
  \item At the moment the estimation process is only capable of two different optimization algorithms:
  
  \medskip
    \begin{itemize}\setlength\itemsep{1em}
    \item Broyden Fletcher Goldfarb Shanno (BFGS) algorithm
    \item  Powell's conjugate direction method
\end{itemize}
\end{itemize}
\end{itemize}

\end{frame}

\begin{frame}
\begin{itemize}\setlength\itemsep{1em}
\item There are two different options for the start values that could be set in the initialization file:\medskip
  \begin{itemize}\setlength\itemsep{1em}
  \item \textit{init:} The estimation process uses the coefficient values specified in the initialization file as the start values for the estimation process.
  \item \textit{auto:} The start values are determined via a simple OLS followed by a Probit regression for the choice indicator.
  \end{itemize}
  \item The estimation results are printed to an \href{examples/est.grmpy.info}{output file}.
\end{itemize}
\end{frame}

\begin{frame}
\textbf{Test battery}

\medskip
\begin{itemize}\setlength\itemsep{1em}
\item We also provide a test battery that includes several tests to ensure that the processes perform as intended.\medskip
\begin{itemize}\setlength\itemsep{1em}
\item Property-based testing
\item Reliability testing
\item Regression testing
\end{itemize}
\end{itemize}
\end{frame}

\begin{frame}
\begin{center}
\Large{\textbf{What's new?}}
\end{center}
\end{frame}

\begin{frame}

  \begin{figure}
  	\caption{Replication Carneiro (2011)}
    \includegraphics[scale=0.5]{figures/fig-marginal-benefit-parametric-replication.png}
  \end{figure}
\end{frame}

\begin{frame}
\begin{figure}
  \caption{Performance comparison}
  \includegraphics[scale=0.5]{figures/fig1.png}
\end{figure}


\end{frame}

\begin{frame}
\textbf{Implementation of standard errors and adjustments on the output files}
\vfill
\begin{center}
See: \href{examples/est.grmpy.info}{Example}
\end{center}
\vfill
\end{frame}

\begin{frame}
\textbf{Online documentation}
\vfill
\begin{figure}
  \includegraphics[scale=0.3]{figures/docu.png}
\end{figure}
\vfill
\end{frame}
